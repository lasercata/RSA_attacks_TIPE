\documentclass[a4paper, 12pt, twoside]{article}


%------------------------------------------------------------------------
%
% Author                :   Lasercata
% Last modification     :   2023.05.21
%
%------------------------------------------------------------------------


%------ini
\usepackage[utf8]{inputenc}
\usepackage[T1]{fontenc}
%\usepackage[french]{babel}
%\usepackage[english]{babel}


%------geometry
\usepackage[textheight=700pt, textwidth=500pt]{geometry}


%------color
\usepackage{xcolor}
\definecolor{ff4500}{HTML}{ff4500}
\definecolor{00f}{HTML}{0000ff}
\definecolor{0ff}{HTML}{00ffff}
\definecolor{656565}{HTML}{656565}

\renewcommand{\emph}{\textcolor{ff4500}}
\renewcommand{\em}{\color{ff4500}}

\newcommand{\strong}[1]{\textcolor{ff4500}{\bf #1}}
\newcommand{\st}{\color{ff4500}\bf}


%------Code highlighting
%---listings
\usepackage{listings}

\definecolor{cbg}{HTML}{272822}
\definecolor{cfg}{HTML}{ececec}
\definecolor{ccomment}{HTML}{686c58}
\definecolor{ckw}{HTML}{f92672}
\definecolor{cstring}{HTML}{e6db72}
\definecolor{cstringlight}{HTML}{98980f}
\definecolor{lightwhite}{HTML}{fafafa}

\lstdefinestyle{DarkCodeStyle}{
    backgroundcolor=\color{cbg},
    commentstyle=\itshape\color{ccomment},
    keywordstyle=\color{ckw},
    numberstyle=\tiny\color{cbg},
    stringstyle=\color{cstring},
    basicstyle=\ttfamily\footnotesize\color{cfg},
    breakatwhitespace=false,
    breaklines=true,
    captionpos=b,
    keepspaces=true,
    numbers=left,
    numbersep=5pt,
    showspaces=false,
    showstringspaces=false,
    showtabs=false,
    tabsize=4,
    xleftmargin=\leftskip
}

\lstdefinestyle{LightCodeStyle}{
    backgroundcolor=\color{lightwhite},
    commentstyle=\itshape\color{ccomment},
    keywordstyle=\color{ckw},
    numberstyle=\tiny\color{cbg},
    stringstyle=\color{cstringlight},
    basicstyle=\ttfamily\footnotesize\color{cbg},
    breakatwhitespace=false,
    breaklines=true,
    captionpos=b,
    keepspaces=true,
    numbers=left,
    numbersep=10pt,
    showspaces=false,
    showstringspaces=false,
    showtabs=false,
    tabsize=4,
    frame=L,
    xleftmargin=\leftskip
}

%\lstset{style=DarkCodeStyle}
\lstset{style=LightCodeStyle}
%Usage : \begin{lstlisting}[language=Caml] ... \end{lstlisting}

%---tcolorbox
\usepackage[many]{tcolorbox}
\DeclareTColorBox{pseudocode}{O{black}O{lightwhite}}{
    breakable,
    outer arc=0pt,
    arc=0pt,
    top=0pt,
    toprule=-.5pt,
    right=0pt,
    rightrule=-.5pt,
    bottom=0pt,
    bottomrule=-.5pt,
    colframe=#1,
    colback=#2,
    enlarge left by=10pt,
    width=\linewidth-\leftskip-10pt,
}


%-------make the table of content clickable
\usepackage{hyperref}
\hypersetup{
    colorlinks,
    citecolor=black,
    filecolor=black,
    linkcolor=black,
    urlcolor=black
}
\usepackage{xurl}
%Uncomment this and comment above for dark mode
% \hypersetup{
%     colorlinks,
%     citecolor=white,
%     filecolor=white,
%     linkcolor=white,
%     urlcolor=white
% }


%------pictures
\usepackage{graphicx}
%\usepackage{wrapfig}

\usepackage{tikz}
\usetikzlibrary{shapes.geometric}


%------tabular
%\usepackage{color}
%\usepackage{colortbl}
%\usepackage{multirow}


%------Physics
%---Packages
%\usepackage[version=4]{mhchem} %$\ce{NO4^2-}$

%---Commands
\newcommand{\link}[2]{\mathrm{#1} \! - \! \mathrm{#2}}
\newcommand{\pt}[1]{\cdot 10^{#1}} % Power of ten
\newcommand{\dt}[2][t]{\dfrac{\mathrm d #2}{\mathrm d #1}} % Derivative


%------math
%---Packages
%\usepackage{textcomp}
%\usepackage{amsmath}
\usepackage{amssymb}
\usepackage{mathtools} % For abs
\usepackage{stmaryrd} %for \llbracket and \rrbracket
\usepackage{mathrsfs} %for \mathscr{x} (different from \mathcal{x})

%---Commands
%-Sets
\newcommand{\N}{\mathbb{N}} %set N
\newcommand{\Z}{\mathbb{Z}} %set Z
\newcommand{\Q}{\mathbb{Q}} %set Q
\newcommand{\R}{\mathbb{R}} %set R
\newcommand{\C}{\mathbb{C}} %set C
\newcommand{\U}{\mathbb{U}} %set U
\newcommand{\seg}[2]{\left[ #1\ ;\ #2 \right]}
\newcommand{\nset}[2]{\left\llbracket #1\ ;\ #2 \right\rrbracket}

%-Exponantial / complexs
\newcommand{\e}{\mathrm{e}}
\newcommand{\cj}[1]{\overline{#1}} %overline for the conjugate.

%-Vectors
\newcommand{\vect}{\overrightarrow}
\newcommand{\veco}[3]{\displaystyle \vect{#1}\binom{#2}{#3}} %vector + coord

%-Limits
\newcommand{\lm}[2][{}]{\lim\limits_{\substack{#2 \\ #1}}} %$\lm{x \to a} f$ or $\lm[x < a]{x \to a} f$
\newcommand{\Lm}[3][{}]{\lm[#1]{#2} \left( #3 \right)} %$\Lm{x \to a}{f}$ or $\Lm[x < a]{x \to a}{f}$
\newcommand{\tendsto}[1]{\xrightarrow[#1]{}}

%-Integral
\newcommand{\dint}[4][x]{\displaystyle \int_{#2}^{#3} #4 \mathrm{d} #1} %$\dint{a}{b}{f(x)}$ or $\dint[t]{a}{b}{f(t)}$

%-left right
\newcommand{\lr}[1]{\left( #1 \right)}
\newcommand{\lrb}[1]{\left[ #1 \right]}
\newcommand{\set}[1]{\left\{ #1 \right\}}
\newcommand{\abs}[1]{\left\lvert #1 \right\rvert}
\newcommand{\ceil}[1]{\left\lceil #1 \right\rceil}
\newcommand{\floor}[1]{\left\lfloor #1 \right\rfloor}
\newcommand{\lrangle}[1]{\left\langle #1 \right\rangle}

%-Others
\newcommand{\para}{\ /\!/\ } %//
\newcommand{\ssi}{\ \Leftrightarrow \ }
\newcommand{\eqsys}[2]{\begin{cases} #1 \\ #2 \end{cases}}

\newcommand{\med}[2]{\mathrm{med} \left[ #1\ ;\ #2 \right]}  %$\med{A}{B} -> med[A ; B]$
\newcommand{\Circ}[2]{\mathscr{C}_{#1, #2}}

\renewcommand{\le}{\leqslant}
\renewcommand{\ge}{\geqslant}

\newcommand{\oboxed}[1]{\textcolor{ff4500}{\boxed{\textcolor{black}{#1}}}} %orange boxed


%------commands
%---to quote french text
\newcommand{\simplecit}[1]{\guillemotleft$\;$#1$\;$\guillemotright}
\newcommand{\cit}[1]{\simplecit{\textcolor{656565}{#1}}}
\newcommand{\quo}[1]{\cit{\it #1}}

%---to indent
\newcommand{\ind}[1][20pt]{\advance\leftskip + #1}
\newcommand{\deind}[1][20pt]{\advance\leftskip - #1}

%---to indent a text
\newcommand{\indented}[2][20pt]{\par \ind[#1] #2 \par \deind[#1]}
\newenvironment{indt}[2][20pt]{#2 \par \ind[#1]}{\par \deind} %Titled indented env

%---title
\newcommand{\thetitle}[2]{\begin{center}\textbf{{\LARGE \underline{\emph{#1} :}} {\Large #2}}\end{center}}


%------Sections
% To change section numbering :
% \renewcommand\thesection{\Roman{section}}
% \renewcommand\thesubsection{\arabic{subsection}}
% \renewcommand\thesubsubsection{\aleph{subsection}}

% To start numbering from 0
% \setcounter{section}{-1}


%------page style
\usepackage{fancyhdr}
\usepackage{lastpage}

\setlength{\headheight}{18pt}
\setlength{\footskip}{50pt}

\pagestyle{fancy}
\fancyhf{}
\fancyhead[LE, RO]{\textit{\textcolor{black}{\today}}}
\fancyhead[RE, LO]{\large{\textsl{\emph{\texttt{\jobname}}}}}

\fancyfoot[RO, LE]{\textit{\texttt{\textcolor{black}{Page \thepage /}\pageref{LastPage}}}} %Change 'black' to 'white' for dark mode
\fancyfoot[LO, RE]{\includegraphics[scale=0.12]{/home/lasercata/Pictures/1.images_profil/logo/mieux/lasercata_logo_fly_fond_blanc.png}}

% For dark mode :
%/home/lasercata/Pictures/1.images_profil/logo/mieux/lasercata_logo_fly.png


%------init lengths
\setlength{\parindent}{0pt} %To avoid using \noindent everywhere.
\setlength{\parskip}{3pt}


%---------------------------------Begin Document
\begin{document}
    
    %For dark mode :
    % \pagecolor{black}
    % \color{white}
    
    \thetitle{TIPE}{Draft}
    
    \tableofcontents
    \newpage
    
    
    \begin{indt}{\section{Common modulus}}
        
        Given : $N$ (common modulus), $e, d$ (known public and private exponent), $e_1$ (public exponent linked to searched private one).
        
        First search $p, q \in \mathbb P\ |\ N = pq$.
        
        We have 
            \[ ed \equiv 1 \ [\phi(N)] \]
            \[ \gcd(e, \phi(N)) = 1 \]
        
        
        Let $k = ed - 1$. By definition, $\exists \lambda \in \N\ |\ ed - 1 = \lambda \cdot \phi(N)$.
        
        However $\phi(N) = (p - 1)(q - 1)$, so $2^2\ |\ \phi(N)$ (because $p, q \in \mathbb P$, thus $p \equiv q \equiv 1 \ [2]$)

        \vspace{12pt}

        Let $n = pq$, and $x \in \Z/n\Z$.
            \[
                \begin{array}{rcl}
                    x \in (\Z/n\Z)^* &\ssi& \exists y \in \Z/n\Z \ |\ xy \equiv 1\ [n]
                    \vspace{6pt}
                    \\
                    &\ssi& \exists (y, k) \in \Z/n\Z \times \N\ |\ xy - kn = 1
                    \vspace{6pt}
                    \\
                    &\ssi& \gcd(x, n) = 1
                \end{array}
            \]
        
    \end{indt}

    \vspace{12pt}
    
    \begin{indt}{\section{Fermat factorisation}}
        if $n$ is composite,
        \[
            \exists a, b \in \N\ |\ n = a^2 - b^2 = (a - b)(a + b) = pq \quad (*)
        \]
        So we have $b^2 = a^2 - n$.

        Then choose $a = \ceil{\sqrt{n}\ }$. If $a^2 - n$ is a square, won. Otherwise, increment $a$.

        \vspace{12pt}
        
        Proof for $(*)$ :

        If $n = pq$, then we have :
        \[
            \lr{\dfrac{p + q}{2}}^2 - \lr{\dfrac{p - q}{2}}^2
            = \dfrac{p^2 + 2pq + q^2}{4} - \dfrac{p^2 - 2pq + q^2}{4}
            = pq = n\ \blacksquare
        \]
    \end{indt}

    \vspace{12pt}
    
    \begin{indt}{\section{Same message}}
        \begin{indt}{\subsection{same modulus}}
            If the same message $m$ is sent to Alice $(e_1, n)$ and Bob $(e_2, n)$.

            Alice receives $c_1 = m^{e_1} \mod n$

            Bob receives $c_2 = m^{e_2} \mod n$.

            We have :
            \[
                c_1 c_2 \equiv m^{e_1} m^{e_2} \equiv m^{e_1 + e_2}\ [n]
            \]
            (pointless)

            If $\gcd(e_1, e_2) = 1$,
            \[
                \exists a, b \in \Z\ |\ ae_1 + be_2 = 1
            \]

            and
            \[
                c_1^a \cdot c_2^b
                \equiv m^{a \cdot e_1} \cdot m^{b \cdot e_2}
                \equiv m^{a e_1 + be_2}
                \equiv m\ [n]
            \]

            But not a realistic situation : Alice ans Bob can calculate each other's private exponent.
        \end{indt}

        \vspace{12pt}
        
        \begin{indt}{\subsection{same message}}
            From \url{https://security.stackexchange.com/questions/166370/how-to-do-rsa-same-message-attack}

            Let be $m$ the message, and $\forall i \in \nset 1 3,\ n_i$ the modulus.

            The keys are $(e, n_i)$, where $e = 3$.

            The encrypted messages are :
            \[
                \forall i \in \nset 1 3,\ c_i \equiv m^e\ [n_i]
            \]

            We have the following system (we need $e$ equations in fact) :
            \[
                \begin{cases}
                    c_1 \equiv m^e\ [n_1]
                    \\
                    c_2 \equiv m^e\ [n_2]
                    \\
                    c_3 \equiv m^e\ [n_3]
                \end{cases}
            \]

            CRT (why ?)

            \[
                M = \prod_{k = 1}^e n_k
            \]
            \[
                \forall k \in \nset 1 e,\ M_k = \dfrac M {n_k}
            \]
            \[
                m^3 \equiv \lr{\sum_{k = 1}^e c_k \cdot M_k \cdot (M_k^{-1} \mod m_i)} \mod M
            \]
        \end{indt}
    \end{indt}

    \vspace{12pt}
    
    \begin{indt}{\section{Large numbers}}
        \begin{indt}{\subsection{Large message}}
            \label{largeMessage}

            Let $m$ be a message, $c = m^e\ [N]$.

            If $m$ is close of $N$ (\textit{cf} paper on large private exponent)

            We have
            \[
                \begin{array}{rcl}
                    N - N^{\tfrac 1 e} < m < N
                    &\ssi& 0 < \underbrace{N - m}_{m_0} < N^{\tfrac 1 e}
                    \\
                    &\ssi& 0 < {m_0}^e < N
                \end{array}
            \]

            So if $N - N^{\tfrac 1 e} < m < N$, let $m_0 = N - m$, and we have
            \[
                m_0 \equiv N - m \equiv -m\ [N]
            \]

            So as $e \equiv 1\ [2]$,
            \[
                m_0^e \equiv (-m)^e \equiv (-1)^e m^e \equiv -m^e \equiv -c\ [N]
            \]

            Thus we can calculate $m_0$ :
            \[
                m_0 = \lr{-c \mod N}^{\tfrac 1 e}
            \]

            And we can recover $m$ :
            \[
                m = N - m_0 = N - \lr{-c \mod N}^{\tfrac 1 e}
            \]
        \end{indt}
    \end{indt}

    \vspace{12pt}
    
    \begin{indt}{\section{Hastad Attack}}
        \begin{indt}{\subsection{Notations}}
            Let $m$ be the message, $p$ the number of messages, $(e_k)_{k \in \nset 1 p}$ the public exponents, and $(n_k)_{k \in \nset 1 p}$ the modulus.
            
            Let $(c_k)_{k \in \nset 1 p}$ the corresponding cipher texts. We have :
            
            \[
                \left\{
                \begin{array}{rcl}
                    c_1 &\equiv& m^{e_1}\ [n_1]
                    \\
                        &\vdots
                    \\
                    c_k &\equiv& m^{e_k}\ [n_k]
                    \\
                        &\vdots
                    \\
                    c_p &\equiv& m^{e_p}\ [n_p]
                \end{array}
                \right.
            \]
            
            We suppose that
            \[
                \forall (i, j) \in \nset 1 p ^2,\
                i \neq j \Rightarrow n_i \wedge n_j = 1
            \]
            
            Otherwise, one can factor the modulus by calculating the gcds.
        \end{indt}

        \begin{indt}{\subsection{CRT}}
            \begin{pseudocode}
                CRT :
            
                Let $n \in \N\ |\ n \ge 2$, and $(a_k)_{k \in \nset 1 n} \subset \N^* \setminus \set 1$ such that
                \[
                    \forall (i, j) \in \nset 1 n ^2,\
                    i \neq j \Rightarrow a_i \wedge a_j = 1
                \]
            
                Then, with $\displaystyle a = \prod_{k = 1}^n a_k$ :
            
                \[
                    \begin{array}{ccccc}
                        \varphi & : & \Z/a\Z & \longrightarrow & \displaystyle \prod_{k = 1}^n \Z / a_k \Z
                        \vspace{3pt}
                        \\
                                &   & cl_a(k) & \longmapsto & (cl_{a_1}(k), \ldots, cl_{a_n}(k))
                    \end{array}
                \]
            
                is a bijection (even a ring isomorphism).
            
                \vspace{12pt}
            
                Determination of $\varphi^{-1}$ :
            
                \[
                    \begin{array}{cl}
                        & \varphi^{-1}((cl_{a_1}(\alpha_1), \ldots, cl_{a_n}(\alpha_n)))
                        \\
                        =& \displaystyle
                        \varphi^{-1}\!\lr{\sum_{k = 1}^n \alpha_k \lr{cl_{a_1}(0), \ldots, cl_{a_{k - 1}}(0), cl_{a_k}(1), cl_{a_{k + 1}}(0), \ldots, cl_{a_n}(0)}}
                        \\
                        =& \displaystyle
                        \sum_{k = 1}^n \alpha_k \underbrace{\varphi^{-1} \lr{cl_{a_1}(0), \ldots, cl_{a_{k - 1}}(0), cl_{a_k}(1), cl_{a_{k + 1}}(0), \ldots, cl_{a_n}(0)}}_{cl_a(m_k)}
                        \\
                        =& \displaystyle \sum_{k = 0}^n \alpha_k cl_a(m_k)
                    \end{array}
                \]
            
                Is is enough to find suitable $m_k$, \textit{i.e} such that $\forall k \in \nset 1 n$,
                \[
                    \begin{cases}
                        m_k \in \Z
                        \\
                        \forall i \in \nset 1 n \setminus \set k,\
                        m_k \equiv 0\ [a_i]
                        \\
                        m_k \equiv 1\ [a_k]
                    \end{cases}
                \]
            
                Let $A = \displaystyle \prod_{k = 1}^n a_k$, and $\forall k \in \nset 1 n$, $A_k = \dfrac{A}{a_k}$
            
                As the $a_k$ are pairwise coprime, $\forall k \in \nset 1 n,\ A_k \wedge a_k = 1$, so using \textsc{Bézout}'s identity :
                \[
                    \exists B_k, b_k \in \Z\ |\
                    A_k B_k + a_k b_k = 1
                \]
                ($B_k \equiv \lr{A_k}^{-1}\ [a_k]$)
            
                \vspace{6pt}
            
                Let $\forall k \in \nset 1 n,\ m_k = A_k B_k \in \Z$.
            
                \vspace{6pt}
            
                We have, $\forall k \in \nset 1 n$ :
                \[
                    m_k \equiv A_k B_k \equiv 1 - a_k b_k \equiv 1\ [a_k]
                \]
            
                and $\forall i \in \nset 1 n \setminus \set k$ :
                \[
                    m_k \equiv A_k B_k \equiv 0\ [a_i]
                \]
                because $a_i | A_k$.
            
                \vspace{12pt}
            
                And finally :
                \[
                    \varphi^{-1}\lr{cl_{a_1}(\alpha_1), \ldots, cl_{a_n}(\alpha_n)}
                    =
                    \sum_{k = 1}^n \alpha_k cl_a(A_k B_k)
                \]
            \end{pseudocode}
            
            \vspace{12pt}
            
            So applied here :
            
            %Let $\forall k \in \nset 1 n,\ C_k \in \Z\ |\ C_k^{e_k} \equiv c_k\ [n_k]$.
            
            Suppose that all $e_k$ are equal to $e \in \Z$.
            
            Then by the previous thing, there is one solution to the system, which is :
            \[
                m^e \equiv \sum_{k = 1}^p c_k N_k M_k\ [N]
            \]
            where $\forall k \in \nset 1 p$ :
            \[
                \begin{array}{rcl}
                    N &=& \displaystyle
                    \prod_{i = 1}^p n_i
                    \vspace{6pt}
                    \\
                    N_k &=& \displaystyle
                    \prod_{\substack{i = 1 \\ i \neq k}}^p n_i
                    = \dfrac N {n_k}
                    \vspace{6pt}
                    \\
                    M_k &\equiv& \lr{N_k}^{-1}\ [n_k]
                \end{array}
            \]
            
            Then we can recover $m$ by calculating the $e$\textsuperscript{th} root of $m^e$ (if $m^e < N$).
        \end{indt}

        \vspace{12pt}

        \begin{indt}{\subsection{Number of equations needed}}
            $\bullet$ Finding the minimal number of equations needed :
            
            If we suppose that all moduli are of the same size (approximatively the same number of bits), we have :
            \[
                \displaystyle m^e < N
                \Rightarrow
                m < \sqrt[e]{N}
                = \prod_{k = 1}^p \sqrt[e]{n_k}
                \approx {\sqrt[e]{n_1}}^p
                = {n_1}^{\tfrac p e}
            \]
            
            So if $p < e$, then $\dfrac p e < 1$, and ${n_1}^{\tfrac p e} < n_1$.
            
            But messages can be up to $n_1$ long, so we need to have $p \le e$.
            
            ---
            
            Example with $e = 3$ : if we have only two equations, then if
            \[
                m \ge n^{\tfrac 2 3} = \dfrac{n}{\sqrt[3]{n}}
            \]
            then we won't be able to compute the $e$\textsuperscript{th}.
            
            ---
            
            \vspace{6pt}
            
            Let note $s$ the bit size of the modulus (often 2048), so $n_1 \approx 2^s = n$.
            
            With a message $m$, the minimum number of equations $p$ that are needed is such that :
            \[
                \begin{array}{rcl}
                    && m < \lr{2^s}^{\tfrac p e} = 2^{\tfrac{sp} e}
                    \vspace{6pt}
                    \\
                    &\ssi& \dfrac{sp}{e} > \log_2(m)
                    \vspace{6pt}
                    \\
                    &\ssi& \oboxed{p > \dfrac{e}{s} \log_2(m)}
                \end{array}
            \]
            
            Typically, $e = 2^{16} + 1$, $s = 2048$, so $\dfrac e s = \dfrac{2^{16} + 1}{2^{11}} \approx \dfrac{2^{16}}{2^{11}} = 2^5 = 32 \approx \dfrac e s$.
            The number $m$ is the encoded version of the string $m_{\rm s}$.
            If $m_{\rm s}$ is $l$ characters long, then $\oboxed{\log_2(m) \approx \alpha l}$, where $\alpha = 8$.
            
            So in order for the attack to work, we need $p > 256 l$ in this case. This is thus not really realistic ...
            
            We have also $p > \alpha \dfrac e s l$.
            
            If we have $p$ equations, it is possible to recover the message if that one has less than $\dfrac{p s}{\alpha e}$ char.
            
            With $p = 3$, $e = 3$, $s = 2048$, then if $l < 2^{11 - 3} = 256$, we are able to recover $m$.
        \end{indt}

        %---

        %To encrypt, we need that the numeral representation of $m$ to be lesser than $n_k,\ \forall k \in \nset 1 p$.

        %So here, we have :
        %\[
        %    \forall k \in \nset 1 p,\ m \le n_k
        %\]

        %So $m^p \le \displaystyle \prod_{k = 1}^p n_k = N$

        %Let suppose that $m > 1$. So $m \ge 2$, and $m^e \ge 2^e$

        %And we need to have $p \ge e$ to be able to compute the $e$-th root and thus for the attack to work (maybe less if the message is small ?).

        %---

        %\vspace{12pt}
        
        %There is something weird : it works when $p = e$, and not otherwise, but this has not been used here ...

        %Indeed, if we remove or add an equation, then the $m^e$ will be changed, as well as $N$, so why would it be $p = e$ the good one ?

        %\vspace{12pt}
        %
        %$\bullet$ Question : if we have a missing equation ? Is it possible to do something ?

        %\vspace{6pt}
        %
        %Let suppose that $c_p$ is unknown. We have $c_p \in \nset 0 {n_p}$.
        %Then, with the same notations as above, we have :
        %\[
        %    m^e \equiv \sum_{k = 1}^{p - 1} c_k N_k M_k + c_p N_p M_p\ [N]
        %\]

        %and we could test for all values of $c_p$ in $\nset 1{n_p}$ to calculate the $e$\textsuperscript{th} root of $m^e$.

        %We have the following system :
        %\[
        %    \begin{cases}
        %        m^e
        %        \equiv \displaystyle
        %        \sum_{k = 0}^{p - 1} c_k N_k' M_k'\ [N']
        %        \\
        %        m^e \equiv c_p\ [n_p]
        %    \end{cases}
        %\]

        %where $\forall k \in \nset 1 {p - 1}$ :
        %\[
        %    \begin{array}{rcl}
        %        N'
        %        &=& \displaystyle
        %        \prod_{i = 0}^{p - 1} n_i
        %        \vspace{6pt}
        %        \\
        %        N_k'
        %        &=& \dfrac{N'}{n_k}
        %        \vspace{6pt}
        %        \\
        %        M_k'
        %        &\equiv& {N_k'}^{-1}\ [n_k]
        %    \end{array}
        %\]

        %We can thus determine $m^e \mod N'$, but we don't have $c_p$.

        \vspace{12pt}
        
        \begin{indt}{\subsection{Trying with large messages}}
            $\bullet$ If $M - M^{\tfrac 1 e} < m < M$, with $M \approx n^p$, at the first approximation, $m \approx M$, then
            \[
                \dfrac e s \log_2(m)
                \approx \dfrac e s \log_2(n^p)
                = \dfrac e s p s
                = ep
            \]
            
            so we need $e$ times more equations by using the normal way (maybe a bit less because of the approximation).
            
            We can calculate, with the CRT (Hastad attack), $m^e \ [M]$, and then we can recover $m$ using \ref{largeMessage}.
            
            \begin{pseudocode}
                So if we are in the right conditions, and if we need to use $ep$ equations to recover $m$ using the first method, then we will be able to recover $m$ with only $p$ equations using this method.
            \end{pseudocode}
            
            We have $p = \ceil{\dfrac e s \log_2(m)}$. If $\exists p' \in \N\ |\ p = ep'$, we can recover $m$ with only $p'$ equations. Is it possible if we have more than $p'$ equations ? Does this gives a constraint on the message ?
            
            \vspace{6pt}
            
            Is it possible to do something with $m' = M - m$ ?
            With the thing below, we have $\log_2(M) \approx \log_2(m)$, so $m' \approx 0$ (need to go to the next order ?).
            In fact, we have $0 < m' < M^{\tfrac 1 e}$.
            
            \vspace{6pt}
            
            If we have $m'$, we can calculate $p' = \ceil{\dfrac e s \log_2(m')}$.
            Then let $p = ep'$, and let $m$ such that $p = \ceil{\dfrac e s \log_2(m)}$.
            
            We now need to find $M$ such that $m' = M - m$.
            
            But $M$ depends on $p$ !
            
            ---
            
            If we have $m$ and $p$, is it possible to make $m'$ ? We need $\dfrac p e$ equations, but this should be an int. What if it is not ?
            
            %TODO: continue this !
            
            \vspace{6pt}
            
            But we did things the opposite way : in reality we have the message encrypted, we don't know its original length, and we have a certain number of equations.
            
            However, given a message, can we determine the number of equations needed to recover it using this method ?
            
            In order for this method to work, we need to have $p \in \N^*$ such that
            \[
                n^p - n^{\tfrac p e} < m < n^p
            \]
            
            \textit{i.e} such that
            \[
                \begin{array}{rcl}
                    2^{sp} - 2^{\tfrac{sp}{e}} < m < 2^{sp}
                    &\ssi&
                    n^p \lr{1 - n^{p\tfrac{1 - e}{e}}} < m < n^p
                    \\
                    &\ssi&
                \end{array}
            \]
            
            What does it implies on $p$ ?
            
            ---
            
            If $M - M^{\tfrac 1 e} < m < M$, and
            %$\floor{\log_{10}(m)} + 1 = \alpha l = \floor{\dfrac{\log_2(m)}{\log_2(10)}} + 1$
            $\log_2(m) \approx \alpha l$, where $l$ is the length of the message (not encoded
            %: \textit{i.e} it is the number of characters in the string
            ), $\alpha$ depends on the encoding, then
            \[
                \log_2\!\lr{M - M^{\tfrac 1 e}} < \alpha l < \log_2(M) = sp
            \]
            
            And $M - M^{\tfrac 1 e} = 2^{sp} - 2^{\tfrac {sp} e} = 2^{sp}\lr{1 - 2^{sp\tfrac{1 - e}{e}}}$ so
            \[
                \dfrac{sp + \log_2\!\lr{1 - 2^{sp\tfrac{1 - e}{e}}}}{\alpha} \le l \le \dfrac{sp}{\alpha}
            \]
            
            But as $e \ge 3$, we have $-1 \le \dfrac{1 - e}{e} \le - \dfrac 2 3$, so as $sp \gg 1$ ($s = 2048$, $p \in \N^*$),
            \[
                \varepsilon = 2^{sp\tfrac{1 - e}{e}} \approx 0
            \]
            
            and
            \[
                \dfrac{sp}{\alpha} \le l \le \dfrac{sp}{\alpha}
            \]
            
            So $l = \dfrac{sp}{\alpha}$. As $l \in \N$, this is only possible if
            \[
                \alpha \mid sp
            \]
            
            ---
            
            At least, we need to have $sp + \varepsilon \le \log_2(m) \le sp$, \textit{i.e} $\oboxed{\log_2(m) \approx sp = \log_2(M)}$.

            % \vspace{12pt}
            % 
            % ---
            %
            % Testing if it is possible.
            %
            % Let $e = 3$, $s = 2048$, $p = 9$. Is it possible to find a message $m$ such that
            % \[
            %     p = \ceil{\dfrac e s \log_2(m)} \quad ?
            % \]
            %
            % We have :
            % \[
            %     \begin{array}{rcl}
            %         p = \ceil{\dfrac e s \log_2(m)}
            %         &\ssi& p < \dfrac e s \log_2(m) \le p + 1
            %         \vspace{6pt}
            %         \\
            %         &\ssi& \dfrac{ps}{e} < \log_2(m) \le \dfrac{p + 1}{e}s
            %         \vspace{6pt}
            %         \\
            %         &\ssi& 2^{\tfrac{ps}{e}} < m \le 2^{\tfrac{ps}{e} + \tfrac s e}
            %         \vspace{6pt}
            %         \\
            %         &\ssi& m \in \underbrace{\nset{2^{\tfrac{ps}{e}} - 1}{2^{\tfrac{ps}{e} + \tfrac s e}}}_{E}
            %     \end{array}
            % \]
            %
            % We have :
            % \[
            %     \begin{array}{rcl}
            %         \mathrm{card}(E)
            %         &=& 2^{\tfrac{ps}{e} + \tfrac s e} - \lr{2^{\tfrac{ps}{e}} - 1} + 1
            %         \\
            %         &=& 2^{\tfrac{ps}{e}} \lr{2^{\tfrac s e} - 1} + 2
            %     \end{array}
            % \]
        \end{indt}
    \end{indt}

    \vspace{12pt}
    
    \begin{indt}{\section{Wiener's attack}}
        \begin{indt}{\subsection{Classic attack}}
            Let
            $
                \left|
                \begin{array}{l}
                    p, q \in \mathbb{P}\ |\ q < p < 2q
                    \vspace{6pt}
                    \\
                    n = pq
                    \vspace{6pt}
                    \\
                    d \in \left[ 1\ ;\ \frac 1 3 n^{\frac 1 4} \right[ \cap \N
                    \vspace{6pt}
                    \\
                    e \in \nset{1}{\phi(n)}\ |\ ed \equiv 1\ [\phi(n)]
                \end{array}
                \right.
            $.
            Let be $\varphi = \phi(n)$.
            
            Given $(e, n)$, one can efficiently recover $d$.
            
            Proof :
            
            Since $ed \equiv 1 \ [\varphi]$, $\exists k \in \N\ |\ ed - k \varphi = 1$, so :
            \[
                \begin{array}{rc}
                    & \dfrac{ed - k \varphi}{d\varphi} = \dfrac{1}{d\varphi}
                    \vspace{6pt}
                    \\
                    \Rightarrow&
                    \dfrac e \varphi - \dfrac k d = \dfrac 1 {d \varphi}
                    \vspace{6pt}
                    \\
                    \Rightarrow&
                    \abs{\dfrac e \varphi - \dfrac k d} = \dfrac 1 {d \varphi}
                \end{array}
            \]
            
            Hence $\dfrac k d$ is an approximation of $\dfrac e \varphi$.
            
            We can now try to approximate $\varphi$ with $n$ :
            \[
                \varphi = \phi(n) = (p - 1)(q - 1) = n - p - q + 1
            \]
            
            And $p + q - 1 < 3\sqrt n$ : since
            $
                \begin{cases}
                    p < 2q
                    \\
                    q < p
                \end{cases}
            $
            (by hypothesis), we have
            \[
                \begin{cases}
                    p + q < 3q
                    \\
                    q^2 < pq = n
                \end{cases}
                \Rightarrow
                \begin{cases}
                    p + q < 3q
                    \\
                    q < \sqrt n
                \end{cases}
                \Rightarrow
                p + q < 3\sqrt n
                \Rightarrow p + q - 1 < 3\sqrt n
            \]
            So $\abs{n - \varphi} = \abs{p + q - 1} < 3\sqrt n$.
            
            \vspace{12pt}
            
            Then we have :
            
            \[
                \begin{array}{rcl}
                    \abs{\dfrac e n - \dfrac k d}
                    &=& \abs{\dfrac{ed - nk}{nd}}
                    \vspace{6pt}
                    \\
                    &=& \abs{\dfrac{ed - k\varphi + k\varphi - nk}{nd}}
                    \vspace{6pt}
                    \\
                    &=& \abs{\dfrac{1 - k(n - \varphi)}{nd}}
                    \vspace{6pt}
                    \\
                    &<& \dfrac{1 + \abs{k(n - \varphi)}}{\abs{nd}}
                    \vspace{6pt}
                    \\
                    &\le& \abs{\dfrac{k(n - \varphi)}{nd}}
                    \vspace{6pt}
                    \\
                    &\le& \abs{\dfrac{3k\sqrt{n}}{nd}}
                    \vspace{6pt}
                    \\
                    &=& \dfrac{3k}{d\sqrt n}.
                \end{array}
            \]
            
            Then, $k \varphi = ed - 1 < ed$ and $e < \varphi$, so $k < \dfrac{e}{\varphi}d < d$, so :
            \[
                k < d < \tfrac 1 3 n^{\frac 1 4}
                \ \Rightarrow\
                \dfrac k d < 1 < \dfrac{n^{\frac 1 4}}{3d}
            \]
            
            Hence :
            \[
                \begin{array}{rcl}
                    \abs{\dfrac e n - \dfrac k d}
                    &\le&
                    \dfrac k d \dfrac{3}{\sqrt n}
                    \vspace{6pt}
                    \\
                    &\le& \dfrac{n^{\frac 1 4}}{3d} \dfrac{3}{\sqrt n}
                    \vspace{6pt}
                    \\
                    &=& \dfrac{1}{d n^{\frac 1 4}}
                \end{array}
            \]
            
            And :
            \[
                2d^2 < \dfrac 2 3 dn^{\frac 1 4} < dn^{\frac 1 4}
                \ \Rightarrow\
                \dfrac 3 {2d n^{\frac 1 4}} < \dfrac 1 {2d^2}
            \]
            
            Hence :
            \[
                \oboxed{
                \abs{\dfrac e n - \dfrac k d}
                \le \dfrac{1}{dn^{\frac 1 4}}
                \le \dfrac{1}{2d^2}
                }
            \]
            
            So $\dfrac e n$ is an approximation of $\dfrac k d$.
            In fact, all fraction approximating $\dfrac e n$ can be obtained as the convergents of the continued fraction expansion of $\dfrac e n$.
            
            The number of such fractions is bounded by $\log_2(n)$ (Why ???), and $\dfrac k d$ is one of them.
            
            Let $k_i$ and $d_i$ the numerator and denominator of the $i$-th convergent of the expansion of $\dfrac e n$ ($i \in \nset{0}{i_{\rm m}}$).
            
            Now compute, $\forall i \in \nset 0 {i_{\rm m}}$, $\varphi_i = \dfrac{e \cdot d_i - 1}{k_i}$.
            
            We know that :
            \[
                \begin{array}{rl}
                    &
                    \begin{cases}
                        n = pq
                        \\
                        \varphi = (p - 1)(q - 1)
                    \end{cases}
                    \vspace{6pt}
                    \\
                    \Rightarrow
                    &
                    \begin{cases}
                        n = pq
                        \\
                        \varphi = n - p - q + 1
                    \end{cases}
                    \vspace{6pt}
                    \\
                    \Rightarrow
                    &
                    \begin{cases}
                        pq = n
                        \\
                        p + q = n - \varphi + 1
                    \end{cases}
                    \vspace{6pt}
                    \\
                    \Rightarrow
                    &
                    p, q \in \set{x \in \R\ |\ x^2 - (n - \varphi + 1)x + n = 0}
                \end{array}
            \]
            ($p \neq q$, otherwise factoring $n$ is simple ...)
            
            So we can calculate $\forall i \in \nset 0 {i_{\rm m}}$ the roots of $x^2 - (n - \varphi_i + 1)x + n$, and check if they factor $n$.
        \end{indt}

        \vspace{12pt}
        
        \begin{indt}{\subsection{Extension with large private exponent}}
            We use the same notations as above, but we take $d$ satisfying :
            \[
                \sqrt 6 (\varphi - d) < n^{\tfrac 1 4}
            \]

            so
            \[
                \begin{array}{rcl}
                    \sqrt 6 (\varphi - d) < n^{\tfrac 1 4}
                    &\ssi& \sqrt 6 d > \sqrt 6 \varphi - n^{\tfrac 1 4}
                    \\
                    &\ssi& d > \varphi - \dfrac{\sqrt 6}{6} n^{\tfrac 1 4}
                \end{array}
            \]

            So $d \in \left] \varphi - \dfrac{\sqrt 6}{6}n^{\tfrac 1 4}\ ;\ \varphi \right[$.

            \[
                \begin{array}{rcl}
                    \varphi - \dfrac{\sqrt 6} 6 n^{\tfrac 1 4} < d < \varphi
                    &\ssi& -\varphi < -d < \dfrac{\sqrt 6}{6}n^{\tfrac 1 4} - \varphi
                    \vspace{3pt}
                    \\
                    &\ssi& 0 < \varphi - d < \dfrac{\sqrt{6}}{6} n^{\tfrac 1 4}
                \end{array}
            \]

            So let $D = \varphi - d$.

            We have $D < \dfrac{1}{\sqrt 6} n^{\tfrac 1 4}$

            The above proof is still correct for such a $D$ (because $\dfrac{\sqrt 2} 2 < 1$)
        \end{indt}
    \end{indt}
    
    
    
\end{document}
%--------------------------------------------End
